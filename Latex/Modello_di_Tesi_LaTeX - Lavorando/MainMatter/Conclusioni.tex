% ------------------------------------------------------------------------ %
% !TEX encoding = UTF-8 Unicode
% !TEX TS-program = pdflatex
% !TEX root = ../Tesi.tex
% !TEX spellcheck = it-IT
% ------------------------------------------------------------------------ %
%
% ------------------------------------------------------------------------ %
% 	CONCLUSIONI
% ------------------------------------------------------------------------ %
%
\cleardoublepage
%
\phantomsection
%
\addcontentsline{toc}{chapter}{Conclusioni}
%
\chapter*{Conclusioni}
%
\markboth{Conclusioni}{Conclusioni}	% headings
%
\label{cap:conclusioni}
%
% ------------------------------------------------------------------------ %
%
In this thesis I have presented the reason why conical mirrors at grazing incidence are used for X-ray radiation. With a focus on the Montel system. After I have described the implementation method for the python library MONWES that will implement the Montel system into the OASYS Graphical Environment.
The test done against the OASYS, for the element already existing, and the benchmarking with \cite{resta2015nested} shows the correct working of the system. 
\\
Moreover, the simulation done on the Montel, show an intrinsic aberration of this particular kind of mirror combination that can't make a perfect focus of an image. On the contrary, focusing a Beam with the same mirrors but with a KB system it is obtained the ideal situation. This mismatch is due to the fact on the orthogonal disposition of the mirrors. There is no way to dispose the Beam on the mirrors in order to cancel out this aberration.
\\
The code is still under development, they are still missing the Wolter system. It is also possible to improve the library including the angle-dependant reflection factor. In the future, after the integration with OASYS, it can be used to designing beamlines that use Montel system.
%
% ------------------------------------------------------------------------ %