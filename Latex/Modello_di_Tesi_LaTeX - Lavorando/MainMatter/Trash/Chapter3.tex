\chapter{Montel System}
\label{capitolo3}
\thispagestyle{empty}

\vspace{0.5cm}

\section{Montel}

\hspace{10mm} As discussed before KB system have some limitation that can be overcome with a different optical system named "Montel". This geometry bring four important advantages for high-precision focusing: 
\\ i) the optical system is more compact which allow greater working space;
\\ ii) the focal distance of the two mirror are the same, this cancel out  the anamorphotism;
\\ iii) the alineation of the system is easier with respect to the KB system because, in this case, only one thing has to be aligned, on the contrary, in the KB there are two separated mirror that has to be aligned;
\\ iv) the divergence that can be collected is larger which allows for greater flux and/or a lower diffraction limit.

\subsection{Optical Design}

The mirrors used in this Montel configuration are mirror that have a cylindrical shape in one direction and elliptical shape in the other direction. One approach to obtain the Montel system is that to use two pre-figured elliptical mirror and grind the cut site at 45$^\circ $ as shown in figure. After that it place the mirrors together makes a good fit with no gap requiring no contouring of the mirror side. Another way involves diveding pre-figured elliptical mirror into two part that, add them together, can form the Montel system. This approaches is primary driven by the fact that in a conventionally polished mirror, the clear aperture area has the best figure and finish. As such uAs such, using two halves of a prefigured mirror cut in the middle has several advantages- including consistency and economy. There are major challenges
however. First, the mirror surface must be protected against damage and deformation during cutting and subsequent figuring operations. After cutting into two, the cut sites must be treated (e.g., etched) to remove any subsurface damages that could alter a mirror's figure. Then the mating side of one of the mirrors must be contoured and polished such that when it is placed against the partner mirror, it makes a nearly perfect fit with good surface quality all the way to the contact edge.This last two-steps are crucial because if there is a significant gap or if the mirror surfaces in the vicinity of the interface are damaged, a significant part of the incident beam could be lost. As an example, we are developing a pair of Montel mirrors for polychromatic nanofocusing on Sector 33 at APS. This beam line will use 40 mm long elliptical mirrors for nano-focusing a 100 $\mu m$ beam to a 50 nm spot at 2000x demagnification. This concave elliptical mirror has a maximum depression of about 6 $\mu m$ at its center. If cut flat and placed against its mating mirror, a gap as large as 6 $\mu m$ is created which loses about 10$\% $ of the 100 $\mu m$ incident beam. Similarly, if the mirror surfaces near the intersection are damaged, then beam loss can be significant.