\chapter{Focusing for X-rays}
\label{Introduzione}
\thispagestyle{empty}


Image formation usually implies some form of focusing. The focusing property depend on the environment in which the radiation its surrounded. In case of the visible light the laws of about lenses and mirrors it's well-known and studied, that is similar for the electron, where the optical element are substituted with electric and magnetic fields to curve the path of them. To study the focusing property of the X-ray radiation, it have to consider the interaction that acts between the radiation and the matter. There are three phenomena that rule this interaction:
\begin{enumerate}
\item elastic scattering;
\item inelastic scattering;
\item absorption via photoelectric effect.
\end{enumerate}
The first effect, where there is an exchange of energy, is constituted by: Thomson scattering, that it is the scattering of electromagnetic radiation by a free non relativistic charged particle \cite{ThomsonScattering}, and Rayleigh scattering, an elastic scattering between the radiation and the strogly bounded electrons that act cooperativley \cite{RayleighScattering}. For the elastic scattering it have to consider also the interference figure that are generated by the interaction between the incident and the scattered radiation (Bragg diffraction), with a defined phase relation. The second effect ,inelastic scattering, or Compton scattering \cite{ComptonScattering}, that occurs when an electron lost by the atom interact with the radiation and absorb a small energy from the X-ray radiation. This scattering is an incoherent effect so there isn't any phase relation between incident and scattered radiation, moreover the atom pass to another quantum state due to the energy absorbed by the electron. The last effect, absorption via photoelectric effect, occur when an bounded electron with an atom get the necessary energy to break the bound and become free (ionization process).
%
\begin{figure}[]
%
\centering
%
\subfloat[][ \label{fig: Scattering1}]
   {\includegraphics[width=.65\textwidth]{Immagini/Chapter1/Scattering1}}\quad
%
\subfloat[][ \label{fig: Scattering2}]
   {\includegraphics[width=.65\textwidth]{Immagini/Chapter1/Scattering2}}
%
\caption{Attenuation coefficient for X-ray radiation of carbon \ref{fig: Scattering1}, and gold \ref{fig: Scattering2}}
\label{fig :AttenuationCoefficient}
%
\end{figure}
Figure \ref{fig :AttenuationCoefficient} show the contribution for the attenuation coefficient, of the different absorption of two different material: carbon (Figure \ref{fig: Scattering1}), and carbon (\ref{fig: Scattering2}).
\\
The X-ray radiation become relevant with the spreads with the synchrotron, and the introduction of plasma sources, because, before that the X-ray source were not sufficiently intense for experiments.

\section{Interaction with Matter}
\label{sec: Interaction with Matter}
Interaction between radiation and matter can be compressed in an coefficient (absorption coefficient)), that rule the attenuation of an incident radiation

\begin{equation}
I = I_0 exp(-\alpha x)
\label{eq: intensity}
\end{equation}

\begin{flushleft}
where $x $ is the thickness of the material, $\alpha$ is the absorption coefficient, and $I_0$ the initial intensity of the beam corresponding to the intensity at $x=0$. Considering the beam as   a plane wave, it is possible to express the  amplitude of the electromagnetic wave as:
\end{flushleft}

\begin{equation}
A=A_0exp(\frac{-2 \pi \beta x}{\lambda})exp(\frac{-2 \pi i (nx-ct)}{\lambda})
\label{eq: amplitude}
\end{equation}

\begin{flushleft}
where $x$ is the position of the front wave, $\lambda$ correspond to the wavelength of the wave in the vacuum, $n $ is a number that correspond to the refractive index and depends from the material, and $\beta $m in this case, correspond to the absorption coefficient of the material. The propagation of the radiation depend from the complex refractive index $n $, that can be expressed as: 
\end{flushleft}

\begin{equation}
\overline{n} = n - i \beta
\label{eq: n_comlex}
\end{equation}

\begin{flushleft}
For X-rays process , the absorption term is the leading term, this mean that the $\alpha$ coefficient can be defined as linearly dependent from the absorption coefficient, where:
\end{flushleft}

\begin{equation}
\alpha = \frac{4 \pi \beta}{\lambda}
\label{eq: alpha1}
\end{equation}

\begin{flushleft}
As it is reported in Appendix A, the absorption values tabulated are given are the mass absorption coefficients $\mu$, where
\end{flushleft}

\begin{equation}
\alpha = \mu \rho
\label{eq: alpha2}
\end{equation}

\begin{flushleft}
where $\rho $ is the density of the material. The mass absorption of a compound is given by
\end{flushleft}

\begin{equation}
\mu \textsubscript{com} = \sum_{j} w_j \mu_j
\label{eq: mu_com}
\end{equation}

\begin{flushleft}
where $\mu_j$ is the mass absorption of a particular element (the ones reported in Appendix A), and $w_j $ is the fraction of the $j $ element in the material. The relation between the absorption coefficient of the material and the mass absorption coefficient is:
\end{flushleft}

\begin{equation}
\alpha \textsubscript{com} = \mu \textsubscript{com} \rho \textsubscript{com}
\label{eq: alpha com}
\end{equation}

\begin{flushleft}
where $\rho \textsubscript{com}$ is the density of the compound.
\end{flushleft}

Because of the dominant energy of the radiation with respect to the matter energies involved in the interaction (X-rays energies spreads from 100eV, soft X-ray, to 10keV, hard X-ray, binding and molecular energies are of the order of few eV), the ionazation process is the leading process in the absorption coefficient. In this case the greater part of the energies involved is transferred to the kinetic term of the ionized electrons. Electron in atom have a well-defined state of energies, so, to be absorbed, the radiation must have the correct value of the electron state energies, the absorption edges in Figure \ref{fig :AttenuationCoefficient}, correspond to the binding energies of the different electron states. Figure \ref{fig: KLM} show the ionizing process of an X-ray photon, and show qualitatively the different energies of an outer electron with respect to the ionized one. In reality, the edges, are less pronunciation as the ones in figure, due to the finite energy width of the states, and because of the environment effect.
\begin{figure}[]
%
\centering
%
\includegraphics[width=.6\textwidth]{Immagini/Chapter1/KLM}
%
\caption{X-ray ionizing process}
%
\label{fig: KLM}
%
\end{figure}
To understand better the absorption of the X-ray radiation it is reported a brief theoretical treatment of the interaction, because the result are useful for the design of the optical element used for X-rays. The calculation start from the elastic scattering between  X-ray photon against free electron (Thomson scattering). The electro-magnetic radiation is characterized by an electric field with amplitude $A_0$ that accelerate a free electron (of charge $e $ and mass $m_e $) by an amount of $A_0 $($e / m $). A charged particle that is accelerated emits radiation, this change the value of the amplitude of the electric field equal to:   Accelerated charges radiate, the amplitude of the electric vector at a distance $r $ from the charge being

\begin{equation}
A_T(\Phi) = \frac{e}{4 \pi \epsilon_0 c^2 r} a \sin \Phi
\label{eq: At1}
\end{equation}

\begin{flushleft}
where $r $ is the distance between the electric field and the electron, $\Phi $ correspond to the angle between the position vector \textbf{r} and acceleration vector \textbf{a} of the electron. So, replacing a with $A_0 $($e / m $):
\end{flushleft}

\begin{equation}
A_T(\Phi) = A_0 \frac{e^2}{4 \pi \epsilon_0 c^2 r} \sin \Phi
\label{eq: At2}
\end{equation}

\begin{flushleft}
The second step i to go further the Thomson scattering, considering the electron bounded with the atoms. For this purpose the Thomson amplitude $A_T (\Phi) $ is multiplied to a complex number defined as a complex atomic scattering $f = f_1 + if_2 $. Thus:
\end{flushleft}

\begin{equation}
A(\Phi, E) = A_t(\Phi) * f(E) = A_T(\Phi) [f_1(E) + if_2(E)]
\label{eq: A(fi, E)}
\end{equation}

\begin{flushleft}
where the two function $f_1 $ and $f_2 $, depend on the energy of the incident X-ray radiation that, to a first approximation, are independent from the angle between the incident and the scattered radiation $\theta $. This approximation has sense because the typical radiation length ($\sim 0.1-10nm $) is much larger than the typical length of the atomic electronic distribution ($\sim 1-50pm $), the consequence of this approximation is the possibility to consider a phase scattering of the atomic wave function. The values of the two function  $f_1 $ and $f_2 $ are calculated in the relativistic quantum dispersion theory and are given by:
\end{flushleft}

\begin{equation}
f_1(E) = Z + 4 \frac{\epsilon_0 m_e c}{h e^2} \int_{0}^{+ \infty}\frac{W^2 \sigma(W)}{E^2 - W^2} dW - \Delta_{rel}
\label{eq: f1}
\end{equation}

\begin{flushleft}
and
\end{flushleft}

\begin{equation}
f_2(E) = 2 \frac{\epsilon_0 m_e c}{h} E \sigma(E)
\label{eq: f2}
\end{equation}

\begin{flushleft}
In Equation \ref{eq: f1}, the first term correspond to the Thomson scattering, where $Z $ correspond to the atomic number of the atom. To add the angle-dependence of the scattering it is used the factor:
\end{flushleft}

\begin{equation}
f0 = \int_{0}^{+ \infty} U(r) sinc[\frac{4 \pi r}{\lambda} \sin \frac{\theta}{2}] dr
\label{f0}
\end{equation}

\begin{flushleft}
where $U(r) $ represent the radical charge distribution and $sinc(x)$ is the cardinal sine function $ = \frac{\sin x}{x} $. Considering a wavelength $\lambda $ of the order of nanometres, if $\sin \frac{\theta}{2} \leq \frac{\lambda}{2}, f_0=Z $, otherwise for $\sin \frac{\theta}{2}=\lambda$, typically, for most element $ f_0 \simeq 0.9Z $.
\end{flushleft}
In Equation \ref{eq: f1}, the second term (the anomalous dispersion integral), represent the oscillation of the electron after the interaction with the radiation, this can be obtained treating the semi-classically the problem. This approach neglect the damping, so, near the absorption edges $f1 $ is inaccurate. The second term of the Equation \ref{eq: f1}, and in Equation \ref{eq: f2} contain $\sigma $ that is the photo ionazation cross section expressed in $m^2 atom ^{-1} $), a coefficient that is related to the mass absorption coefficient in this way: 

\begin{equation}
\sigma(E) = A \frac{\mu}{N_0}
\label{eq: sigma}
\end{equation}

\begin{flushleft}
where $A $ is the atomic weight and $N_0 $ the Avogadro's number $N_0 = 6.22 * 10^{23} particle*mol^{-1} $. The value of $\sigma(E) $ is theoretically obtained knowing the atomic wave function of the atom, so, only for hydrogen it possible to have the correct value, for all the other system, the calculation can be done with approximation methods that give some uncertainty on $\sigma(E) $, consequently on the value of $f_1 $ and $f_2 $.
\end{flushleft}
In Equation \ref{eq: f1} the third term take in account the relativistic effect. This correction is given by:
\begin{equation}
\Delta_{rel} = \frac{5}{3} \frac{|E_{tot}|}{m_e c^2} + \frac{Z}{2} \left( \frac{E}{m_e c^2} \right)^2
\label{eq: Delta_rel}
\end{equation}

\begin{flushleft}
where $|E_{tot} |$ is the modulus of the total energy of the atom (that is negative), moreover, this third term is the less relevant in Equation \ref{eq: f1}, for X-ray energies, so it is possible to neglect it in the calculation.
\\
For photo absorption event by an electron bounded to an atom, far from the absorption edges, a good approximation is to consider the solid state environment distorted by the ionization of the electrons , because, the most affected electrons are the outer ones. After long calculation, is possible to relate the factors $f_1 $ and $f_2 $ with the macroscopic parameters $n $ and $\beta $:
\end{flushleft}

\begin{equation}
\delta = 1 - n = \frac{e^2 \hbar^2}{2 \epsilon_0 m_e E^2} \overline{f_1}
\label{eq: delta}
\end{equation}

\begin{flushleft}
and
\end{flushleft}

\begin{equation}
\beta = \frac{e^2 \hbar^2}{2 \epsilon_0 m_e E^2} \overline{f_2}
\label{eq: beta}
\end{equation}

\begin{flushleft}
where $\overline{f_1} $ and $\overline{f_2} $ are defined as follow:
\end{flushleft}

\begin{equation}
\overline{f_1} = \sum_j N_j f_{1j} \qquad \overline{f_2} = \sum_j N_j f_{2j}
\label{f1, f2, mean}
\end{equation}

\begin{flushleft}
and represent the average scattering factor per unit volume, $N_j $ is the total number of the particular $j $ element per unit volume. Putting everything together Equation \ref{eq: delta}, apart near the absorption edges, can be expressed as:
\end{flushleft}

\begin{equation}
\delta = \frac{N e^2 \hbar^2}{2 \epsilon_0  E^2} = \frac{N e^2 \lambda^2}{8 \pi^2 \epsilon_0 m_e c^2}
\label{eq: delta new}
\end{equation}

\begin{flushleft}
where $N $ is the number of electrons per unit volume. For X-ray energies the value of $\delta $ is small (typically $\sim 10^{-3} $) and positive, this is important because it means that, for X-rays, the refractive index is a bit less than $1 $. It is possible to find the tabulated values of $f_1 $ and $f_2 $, \cite{henke1981atomic}, that are the main ingredient to calculate the curve in Figure \ref{fig :AttenuationCoefficient}  and these were used to generate Figure 1. This values, according with the experimental results, allow to write, far from absorption edges, the absorption coefficient $\beta $ such as:
\end{flushleft}

\begin{equation}
\beta \sim Z^2 \lambda^3
\label{eq: new beta}
\end{equation}


\section{Total External Reflection}
\label{sec: Total Externa Reflection}
\begin{flushleft}
For the system in Figure \ref{fig: System}, there are two complex refractive index:
\end{flushleft}
\begin{equation}
\overline{n_1} = 1 - \delta_1 - i \beta_1
\label{eq: n1}
\end{equation}
and
\begin{equation}
\overline{n_2} = 1 - \delta_2 - i \beta_2
\label{eq: n1}
\end{equation}
\begin{flushleft}
moreover $\delta_2 > \delta_1 $. In the general case there are, as shown in Figure \ref{fig: System} a reflected and a transmitted wave. For the theoretical treatment, initially, will be neglect the absorption ($\beta_1 = \beta_2 = 0$), moreover the permeability coefficient it is supposed to be similar to the permeability in the vacuum. Thus, the law of Snell, can be expressed such as: 
\end{flushleft}
\begin{equation}
\frac{\cos \theta_i}{\cos \theta_t} = \frac{1 - \delta_2}{1 - \delta_1}
\label{eq: snell 1}
\end{equation}
\begin{figure}[]
%
\centering
%
\includegraphics[width=.6\textwidth]{Immagini/Chapter1/System}
%
\caption{Interface of two medium}
%
\label{fig: System}
%
\end{figure}
\begin{flushleft}
Using the frame system as in Figure \ref{fig: System}, with the z-axis that correspond to the normal of the interface. It is possible to write the component of the electric field of the waves in this way
\end{flushleft}
\begin{subequations}
\begin{equation}
E_{ix} = A_{\parallel} \sin \theta_i \exp^{- i \tau_i}, \hspace{4mm}
E_{iy} = A_{\perp} \exp^{- i \tau_i}, \hspace{4mm} 
E_{iz} = A_{\parallel} \cos \theta_i \exp^{- i \tau_i}
\label{eq: E component 1}
\end{equation}
\begin{equation}
E_{tx} = - T_{\parallel} \sin \theta_t \exp^{- i \tau_t}, \hspace{4mm}
E_{ty} = T_{\perp} \exp^{- i \tau_t}, \hspace{4mm} 
E_{tz} = T_{\parallel} \cos \theta_t \exp^{- i \tau_t}
\label{eq: E component 2}
\end{equation}
\begin{equation}
E_{rx} = R_{\parallel} \sin \theta_r \exp^{- i \tau_r}, \hspace{4mm}
E_{ry} = R_{\perp} \exp^{- i \tau_r}, \hspace{4mm} 
E_{rz} = R_{\parallel} \cos \theta_r \exp^{- i \tau_r}
\label{eq: E component 3}
\end{equation}
\end{subequations}
\noindent  where
\begin{subequations}
\begin{equation}
\tau_i = \omega ( t - \frac{\vec{r} \bullet \vec{s_i} }{v_1}) = \omega \left[ t - \frac{(1 - \delta_1 ) ( x \cos \theta_i + z \sin \theta_i}{c} \right]
\label{eq: tau 1}
\end{equation}
\begin{equation}
\tau_t = \omega ( t - \frac{\vec{r} \bullet \vec{s_t} }{v_2}) = \omega \left[ t - \frac{(1 - \delta_2 ) ( x \cos \theta_t + z \sin \theta_t}{c} \right]
\label{eq: tau 2}
\end{equation}
\begin{equation}
\tau_r = \omega ( t - \frac{ \vec{r} \bullet \vec{s_r} }{v_1}) = \omega \left[ t - \frac{(1 - \delta_1 ) ( x \cos \theta_r + z \sin \theta_r}{c} \right]
\label{eq: tau 3}
\end{equation}
\end{subequations}
\begin{flushleft}
where $\omega $ is the angular frequency of the wave, and $v_1, v_2 $, correspond to the velocities of propagation that depend on the material as follow:
\end{flushleft}
\begin{equation}
v_1 = \frac{c}{1 - \delta_1}, \hspace{4mm} v_2 = \frac{c}{1 - \delta_2}
\end{equation}
\begin{flushleft}
the related magnetic field are:
\end{flushleft}
\begin{subequations}
\begin{equation}
\begin{aligned}
H_{ix} = - A_{\perp} (1 - \delta_1) \sin \theta_i \exp^{-i \tau_i}, \hspace{4mm}
H_{iy} = - A_{\parallel} (1 - \delta_1) \exp^{-i \tau_i}, \\
H_{iz} =  A_{\perp} (1 - \delta_1) \cos \theta_i \exp^{-i \tau_i}
\end{aligned}
\label{eq: H1}
\end{equation}
\begin{equation}
\begin{aligned}
H_{tx} = - T_{\perp} (1 - \delta_2) \sin \theta_t \exp^{-i \tau_t}, \hspace{4mm}
H_{ty} = - T_{\parallel} (1 - \delta_2) \exp^{-i \tau_t}, \\
H_{tz} =  T_{\perp} (1 - \delta_2) \cos \theta_t \exp^{-i \tau_t}
\end{aligned}
\label{eq: H2}
\end{equation}
\begin{equation}
\begin{aligned}
H_{rx} = - R_{\perp} (1 - \delta_1) \sin \theta_r \exp^{-i \tau_r}, \hspace{4mm}
H_{ry} = - R_{\parallel} (1 - \delta_1) \exp^{-i \tau_r}, \\
H_{rz} =  R_{\perp} (1 - \delta_1) \cos \theta_r \exp^{-i \tau_r}
\label{eq: H3}
\end{aligned}
\end{equation}
\end{subequations}
\begin{flushleft}
the boundary condition impose the continuity of the fields:
\end{flushleft}
\begin{equation}
E_{ix} + E_{rx} = E_{tx}, \hspace{4mm} E_{iy} + E_{ry} = E_{ty}
\label{eq: continuity E}
\end{equation}
\noindent and
\begin{equation}
H_{ix} + H_{rx} = H_{tx}, \hspace{4mm} H_{iy} + H_{ry} = H_{ty}
\label{eq: continuity H}
\end{equation}
\begin{flushleft}
because of Snell's laws $\theta_r = \theta_t $, so, from the Equation \ref{eq: continuity E} and Equation \ref{eq: continuity H}:
\end{flushleft}
\begin{subequations}
\begin{equation}
(A_{\parallel} - R_{\parallel}) \sin \theta_i = T_{\parallel} \sin_t
\label{eq: mix1}
\end{equation}
\begin{equation}
A_{\perp} + R_{\perp} = T_{\perp}
\label{eq: mix2}
\end{equation}
\begin{equation}
(1 - \delta_1 ) (A_{\perp} - R_{\perp}) \sin \theta_i = (1 - \delta_2) T_{\perp} \sin \theta_t
\label{eq: mix3}
\end{equation}
\begin{equation}
(1 - \delta_1) (A_{\parallel} +R_{\parallel}) = (1 - \delta_2) T_{\parallel}
\label{eq: mix4}
\end{equation}
\label{eq: mix}
\end{subequations}
\begin{flushleft}
Equations \ref{eq: mix} give a set of equations where the parallel and perpendicular component of the waves are independent. Solving that set with respect to each parallel/perpendicular component it is obtained:
\end{flushleft}
\begin{subequations}
\begin{equation}
\begin{aligned}
\frac{R_{\parallel}}{A_{\parallel}} = \left[\frac{(1 - \delta_2) \sin \theta_i - (1 - \delta_1) \sin \theta_t}{(1 - \delta_2) \sin \theta_i} + (1 - \delta_1) \sin \theta_t \right]
\end{aligned}
\label{eq: R/A parll}
\end{equation}
\begin{equation}
\begin{aligned}
\frac{R_{\perp}}{A_{\perp}} = \left[\frac{(1 - \delta_1) \sin \theta_i - (1 - \delta_2) \sin \theta_t}{(1 - \delta_1) \sin \theta_i} + (1 - \delta_2) \sin \theta_t \right]
\end{aligned}
\label{eq: R/A perp}
\end{equation}
\begin{equation}
\begin{aligned}
\frac{T_{\parallel}}{A_{\parallel}} = \frac{2(1 - \delta_1) \sin \theta_i}{(1 - \delta_2) \sin \theta_i + (1 - \delta_1) \sin \theta_t}
\end{aligned}
\label{eq: T/A parll}
\end{equation}
\begin{equation}
\begin{aligned}
\frac{T_{\perp}}{A_{\perp}} = \frac{2(1 - \delta_1) \sin \theta_i}{(1 - \delta_1) \sin \theta_i + (1 - \delta_2) \sin \theta_t}
\end{aligned}
\label{eq: T/A perp}
\end{equation}
\label{eq: parall and perp 1}
\end{subequations}
\begin{flushleft}
Equations \ref{eq: parall and perp 1} are the Fresnel formula for reflection at a plane surface. Combining them with Equation \ref{eq: snell 1} it is obtained:
\end{flushleft}
\begin{subequations}
\begin{equation}
\begin{aligned}
\frac{R_{\parallel}}{A_{\parallel}} = \frac{(1 - \delta_2)^2 \sin \theta_i - (1 - \delta_1) \sqrt{(1 - \delta_2)^2 - (1 - \delta_1)^2 \cos^2 \theta_i}}{(1 - \delta_2)^2 \sin \theta_i +  (1 - \delta_1) \sqrt{(1 - \delta_2)^2 - (1 - \delta_1)^2 \cos^2 \theta_i}} 
\end{aligned}
\label{eq: R/A parll 1}
\end{equation}
\begin{equation}
\begin{aligned}
\frac{R_{\perp}}{A_{\perp}} = \frac{(1 - \delta_1)^2 \sin \theta_i - \sqrt{(1 - \delta_2)^2 - (1 - \delta_1)^2 \cos^2 \theta_i}}{(1 - \delta_1)^2 \sin \theta_i  +  \sqrt{(1 - \delta_2)^2 - (1 - \delta_1)^2 \cos^2 \theta_i}} 
\end{aligned}
\label{eq: R/A perp 1}
\end{equation}
\begin{equation}
\begin{aligned}
\frac{T_{\parallel}}{A_{\parallel}} = \frac{2(1 - \delta_1) (1 - \delta_2) \sin \theta_i }{(1 - \delta_2)^2 \sin \theta_i  +  (1 - \delta_2)\sqrt{(1 - \delta_2)^2 - (1 - \delta_1)^2 \cos^2 \theta_i}} 
\end{aligned}
\label{eq: T/A parll 1}
\end{equation}
\begin{equation}
\begin{aligned}
\frac{T_{\perp}}{A_{\perp}} = \frac{2(1 - \delta_1) \sin \theta_i }{(1 - \delta_1) \sin \theta_i  +  \sqrt{(1 - \delta_2)^2 - (1 - \delta_1)^2 \cos^2 \theta_i}} 
\end{aligned}
\label{eq: T/A perp 1}
\end{equation}
\label{eq: parall and per 2}
\end{subequations}
\begin{flushleft}
When $\theta_i $ is such that:
\end{flushleft}
\begin{equation}
\cos \theta_i = \frac{1 - \delta_2}{1 - \delta_1}
\label{eq: theta_c}
\end{equation}
\begin{flushleft}
that angle is named critical angle $\theta_c $, and
\end{flushleft}
\begin{equation}
\frac{R_{\parallel}}{A_{\parallel}} = \frac{R_{\perp}}{A_{\perp}}
\label{eq: R/A critical}
\end{equation}
\begin{flushleft}
this case correspond to a wave that is totally reflected. Normally the total external reflection take place at an interface light material(air/vacuum) and dense material, so $\delta_1 = 0, \delta_2 = \delta$, the equations became:
\end{flushleft}
\begin{subequations}
\begin{equation}
\begin{aligned}
\frac{R_{\parallel}}{A_{\parallel}} = \frac{(1 - \delta)^2 \sin \theta_i - \sqrt{(1 - \delta)^2 - \cos^2 \theta_i}}{(1 - \delta)^2 \sin \theta_i + \sqrt{(1 - \delta_2)^2 - \cos^2 \theta_i}} 
\end{aligned}
\label{eq: R/A parll 2}
\end{equation}
\begin{equation}
\begin{aligned}
\frac{R_{\perp}}{A_{\perp}} = \frac{\sin \theta_i - \sqrt{(1 - \delta)^2 - (1 - \cos^2 \theta_i}}{\sin \theta_i  +  \sqrt{(1 - \delta)^2 - \cos^2 \theta_i}} 
\end{aligned}
\label{eq: R/A perp 2}
\end{equation}
\begin{equation}
\begin{aligned}
\frac{T_{\parallel}}{A_{\parallel}} = \frac{2(1 - \delta) \sin \theta_i }{(1 - \delta)^2 \sin \theta_i  +  \sqrt{(1 - \delta)^2 - (1 - \cos^2 \theta_i}} 
\end{aligned}
\label{eq: T/A parll 2}
\end{equation}
\begin{equation}
\begin{aligned}
\frac{T_{\perp}}{A_{\perp}} = \frac{2 \sin \theta_i }{ \sin \theta_i  +  \sqrt{(1 - \delta)^2 - \cos^2 \theta_i}} 
\end{aligned}
\label{eq: T/A perp 2}
\end{equation}
\label{eq: parall and per 3}
\end{subequations}
\begin{flushleft}
introducing the absorbing coefficient $\beta_2 = \beta \neq 0 $:
\end{flushleft}
\begin{subequations}
\begin{equation}
\begin{aligned}
\frac{R_{\parallel}}{A_{\parallel}} = \frac{\overline{n}^2 \sin \theta_i - \sqrt{\overline{n}^2 - \cos^2 \theta_i}}{\overline{n}^2 \sin \theta_i + \sqrt{\overline{n}^2 - \cos^2 \theta_i}} 
\end{aligned}
\label{eq: R/A parll 3}
\end{equation}
\begin{equation}
\begin{aligned}
\frac{R_{\perp}}{A_{\perp}} = \frac{\sin \theta_i - \sqrt{\overline{n}^2 - \cos^2 \theta_i}}{(\sin \theta_i  +  \sqrt{\overline{n}^2 - \cos^2 \theta_i}} 
\end{aligned}
\label{eq: R/A perp 3}
\end{equation}
\begin{equation}
\begin{aligned}
\frac{T_{\parallel}}{A_{\parallel}} = \frac{2\overline{n} \sin \theta_i }{\overline{n}^2 \sin \theta_i  +  \sqrt{\overline{n}^2 - \cos^2 \theta_i}} 
\end{aligned}
\label{eq: T/A parll 3}
\end{equation}
\begin{equation}
\begin{aligned}
\frac{T_{\perp}}{A_{\perp}} = \frac{2 \sin \theta_i }{ \sin \theta_i  +  \sqrt{\overline{n}^2 - \cos^2 \theta_i}} 
\end{aligned}
\label{eq: T/A perp 3}
\end{equation}
\label{eq: parall and per 4}
\end{subequations}
For interface that are curved, the Equations \ref{eq: parall and per 4} are still valid if the curvature radius is much grater that the wavelength, condition that is satisfied for the X-ray radiation.
Let's define the two reflectivity:
\begin{equation}
R_p =\frac{R_{\parallel}}{A_{\parallel}} \left(\frac{R_{\parallel}}{A_{\parallel}} \right)^{*}
\label{eq: Rp}
\end{equation}
\noindent and
\begin{equation}
R_p =\frac{R_{\parallel}}{A_{\parallel}} \left(\frac{R_{\parallel}}{A_{\parallel}} \right)^{*}
\label{eq: Rs}
\end{equation}
\begin{flushleft}
Figure \ref{fig :R_p} show the trend of $R_p $ for two different $\delta $ (Figure \ref{fig: Rp} $\delta = 0.005$, Figure \ref{fig: Rp1} $\delta = 0.001$), changing the absorption coefficient $\beta $. It is possible to note that $R_p $ have a big values for small $\beta $ for small grazing incidence angles, minor than the critical angle, beyond the critical angle, $R_p $ decrease quickly to zero. On the contrary, for big values of $\beta $, $R_p $ decrease gradually before and after the critical angle. The situation of $R_s $, in case of small angle, is similar $R_s \simeq R_p $ as it is showed in Figure \ref{fig :R_s}, where it is plotted the ratio $\frac{R_s}{R_p} $ with respect to the incidence angle. As it is showen, for small angle, the two values are similar, increasing the angle the situation differ, but, in this case, the values of $R_s $ and $R_p $ is very small as it is showed in Figure \ref{fig: Rtot}. In reality $\beta $ is never zero so it is not possible to have a total external reflection, it is convenient to define that the total external reflection occurs when the curve reflectivity with respect to the incidence angle have a point of inflection. Figure \ref{fig: InflectionPoint} show that the condition it is satisfied when: 
\end{flushleft}
\begin{equation}
\beta < 0.63 \delta
\label{eq: last}
\end{equation}
\begin{figure}[]
%
\centering
%
\subfloat[][$\delta = 0.005 $ \label{fig: Rp}]
   {\includegraphics[width=.55\textwidth]{Immagini/Chapter1/Rp}}\quad
%
\subfloat[][$\delta = 0.001 $  \label{fig: Rp1}]
   {\includegraphics[width=.55\textwidth]{Immagini/Chapter1/Rp2}}
%
\caption{$R_p$'s values for grazing incidence with two values of $\delta $, and some values of $\beta $}
\label{fig :R_p}
%
\end{figure}
\begin{figure}[]
%
\centering
%
\subfloat[][$\delta = 0.005 $ \label{fig: Rs}]
   {\includegraphics[width=.65\textwidth]{Immagini/Chapter1/Rs}}\quad
%
\subfloat[][$\delta = 0.001 $  \label{fig: Rs1}]
   {\includegraphics[width=.65\textwidth]{Immagini/Chapter1/Rs2}}
%
\caption{Ratio of $\frac{R_s}{R_p}$ for two values of $\delta $, and some values of $\beta $}
\label{fig :R_s}
%
\end{figure}
\begin{figure}[]
%
\centering
%
\includegraphics[width=.6\textwidth]{Immagini/Chapter1/R_tot}
%
\caption{Reflectivity over the whole values of incidence angles}
%
\label{fig: Rtot}
%
\end{figure}
\begin{figure}[]
%
\centering
%
\includegraphics[width=.7\textwidth]{Immagini/Chapter1/InflectionPoint}
%
\caption{Slopes of reflectivity which show that, to have an inflection point, $\beta < 0.69 \delta$}
%
\label{fig: InflectionPoint}
%
\end{figure}