\chapter{MONWES}
\label{capitolo3}
\thispagestyle{empty}
\vspace{0.5cm}

Chapter \ref{capitolo3} goes deep in the description of the python library developed by me. MONWES (MOntel and Wolter of yunES) library is a python library that perform ray tracing simulations for x-ray radiation along a system consisting of one or several mirrors, using the optical element and compound optical elements introduced in Chapter \ref{capitolo2}. The library take inspiration from OASYS ray tracing software developed by Manuel Sanchez Del Rio of ESRF, taking the starting element already present in OASYS environment in order to have the basic element that will be used in the new Montel system. OASYS elements are written for a sequential tracing method, method that doesn't work for Montel system. Thus a great effort has been done to generate a new tracing system that use a parallel approach.
\\
Object oriented programming were used consisting in a programming method in which data structures are defined with their type of operation, that are applied to the data structure. This oriented programming is based on the concept of "objects", it contains data in form of fields known as attributes and code in the form of procedure known as methods \cite{gamma2001design}.
\\
The final goal of the MONWES library is that to complement the standard ray tracing code SHADOW in OASYS with compound elements.
\\
The library and also this Chapter, is composed by three main elements:
\begin{enumerate}
\item Beam object: containing the information of the beam along the beamline;
\item Optical element: that initialize the characteristic of the optical element/system used in the beamline;
\item Tracing system: that relates the Beam characteristic with the effect of the optical elements/systems.
\end{enumerate}

\vspace{0.5cm}

\section{Beam}
Beam object is the object that contain the spatial and velocity information of a collection of rays. The Beam object is mainly characterized by four parameters: 
\begin{itemize}
\item Number of rays
\item Spatial profile
\item Divergence profile
\item Flag vector
\end{itemize}
Having a large number of rays is useful in terms of results but increase the computation time, so, one has to find a compromise between the number of rays and the computation time depending on the quality desired. By default, the number of rays, is set $25*10^3 $ that allow to have good results without spending lot of time. Figure \ref{fig: BeamRays} show an initialization of Beam setting the number of ray equal to $10^4 $.
\begin{figure}[H]
%
\centering
%
\includegraphics[width=.4\textwidth]{Immagini/Chapter3/BeamRays}
%
\caption{Example of Beam initialization}
%
\label{fig: BeamRays}
%
\end{figure}
The Beam is initialized with its number of rays and the spatial and divergence profile. For the spatial profile there are some possibilities that correspond to different geometrical figure such as: rectangular profile, circular profile, Gaussian profile, point wise profile. The default one is the point wise profile, that it is useful for ideal testing, in this case, there is no input parameter are needed. All the other profiles are parametrized by external input that depends on the nature of the profile. For the Gaussian profile the parameter needed is the $\sigma_x $ and $\sigma_z $ that correspond the $\sigma $ of the two dimension, Figure \ref{fig: CodeGaussian} shows a code example to define a Gaussian spot of a Beam having $25*10^3 $rays with the two $\sigma $ different, $\sigma_x = 0.1mrad, \sigma_z = 1mrad $ as it is showed in Figure \ref{fig: PlotGaussianExample}.
%
\begin{figure}[H]
%
\centering
%
\subfloat[][Example code for a Gaussian source in spatial coordinate\label{fig: CodeGaussian}]
   {\includegraphics[width=.4\textwidth]{Immagini/Chapter3/GaussianSpot}}\quad
%
\subfloat[][Plot of Figure \ref{fig: CodeGaussian}\label{fig: PlotGaussianExample}]
   {\includegraphics[width=.4\textwidth]{Immagini/Chapter3/GaussianExamplePlotSpot}}
%
\caption{Example 1}
\label{fig :p1}
%
\end{figure}
Over the Gaussian profile, there are other two geometrical profile that can be defined: rectangular and circular. They have a uniform distribution of rays in their space domain. For the rectangular distribution the parameter to define are the xz limit of the coordinate that define the sides of the rectangle. Figure \ref{fig: CodeRectangleCircle} shows an example code where it is defined a circular profile with a radius of 1cm and, after, it is overwritten by another geometrical profile having a rectangular shape. In this case the final profile of the Beam is shown in Figure \ref{fig: PlotRectangleExample}.
%
\begin{figure}[H]
%
\centering
%
\subfloat[][Example code for a circular and a rectangular spot\label{fig: CodeRectangleCircle}]
   {\includegraphics[width=.4\textwidth]{Immagini/Chapter3/CodeCircularRectangular}}\quad
%
\subfloat[][Example plot of the rectangular profile\label{fig: PlotRectangleExample}]
   {\includegraphics[width=.4\textwidth]{Immagini/Chapter3/RectangularExamplePlotSpot}}
%
\caption{Example 2}
\label{fig :p2}
\end{figure}
It is possible to define a special shape that have, more or less, the figure of a person with a uniform distribution of the point in all the point of the space. This special shape is showed in Figure \ref{fig: PersonShape}, that is defined in the code written in Figure \ref{fig: CodePerson}. It is used to do a test "imaging" of a system checking that the image, after the optical system, is affected by deformation or defect. As it is showed, the "$initialize\_$ $as\_$ $person$" command take two input parameters, the number of the total rays (by default are $25*10^3 $), and a size parameter that set the coordinate limit of the figure, more precisely. In Figure \ref{fig: PersonShape} the size correspond to $10^{-6} $ so the limit are: $x_{max} =  1\mu m $, $x_{min} = -1\mu m $, $z_{max} = 1\mu m $, $z_{min} = - 20\mu m $.\begin{figure}[H]
%
\centering
%
\subfloat[][Example plot "person" profile\label{fig: PersonShape}]
   {\includegraphics[width=.4\textwidth]{Immagini/Chapter3/PersonExamplePlotSpot}}\quad
%
\subfloat[][Example code for the "person" spot\label{fig: CodePerson}]
   {\includegraphics[width=.4\textwidth]{Immagini/Chapter3/CodePerson}}
%
\caption{Example 3}
\label{fig :p3}
\end{figure}
The last piece of the Beam object is the "Flag" vector. Every component of this vector have a correspondence with a certain ray and contain the information about the number of optical element that, the ray, travel until a particular moment. Moreover, this value become negative when the ray doesn't hit an optical element, in such a way to have an information where the rays were lost. Figure \ref{fig: ResumeBeam}, resume the main parameter of the Beam object with their default values
\begin{figure}[H]
%
\centering
%
\includegraphics[width=.4\textwidth]{Immagini/Chapter3/ResumeBeam}
%
\caption{Summary of the Beam object parameter}
%
\label{fig: ResumeBeam}
%
\end{figure}
A part from the principal characteristic treated above, Beam object contain methods or functions in order to manage better the information that it contain. The other option defined are reported here below:
\begin{itemize}
\item "$import\_$ $from\_$" file(filename='filename'): define a Beam with a characteristic defined in a file '.h5'
\item "$set\_$ $point(x,y,z)$": move the Beam in a part of the space centred in the coordinate (x,y,z)
\item "$initialize\_$ $from\_$ $arrays(x, y, z, vx, vy, vz, flag)$": define a Beam with the spatial value defined in the array x,y,z, the velocities' value defined in the array vx,vy,vz and the flag value defined in the array flag
\item "$duplicate()$": duplicate a Beam
\item "$good\_ beam()$": define a Beam that, starting from another Beam, extract only the good rays (those that have a positive flag)
\item "$part\_$ $of\_$ $beam(indices)$": define a Beam that, starting from another Beam, extract the ray that correspond to the position defined in the array indices
\item "$number\_$ $of\_$ $good\_$ $rays()$": return the values of the good rays
\item "$merge(beam2)$": merge a beam1 with another beam2, the first part of this new beam correspond to the beam1, and the second part to the beam2
\item "$retrace(distance)$": this correspond to a free propagation in the space of the Beam within a distance equal to "distance" 
\end{itemize}
At the end there are the command that plot the various characteristic of the beam, that contain the information for the plotted characteristic, for example plot\_ xy() make a plot of the x and y coordinate of the beam, plot\_ good\_ xpzp() make a plot of the x-velocities and z-velocities of only the rays that have a positive flag


\section{Optical Elements}
Because, as discussed in Chapter \ref{Introduzione}, mirrors are the principal elements used in synchrotron we first develop mirrors in the library. It is also interesting to implement an ideal lens that is useful to verify ideal focusing of the Beam. Only one kind of lens is implemented: ideal lens.
\subsection{Mirrors}
The different kind of mirror that are defined are:
\begin{itemize}
\item plane mirror
\item spherical mirror
\item ellipsoidal mirror
\item paraboloidal mirror
\item hyperboloidal mirror
%
\end{itemize}
All those geometrical shape are a subset of a surface conical figure. As is discussed in Chapter 2, and reported in Equation \ref{eq: SurfaceConic}, a surface conic is defined by a series of coefficient.
\begin{equation}
a_0 x^2 + a_1 y^2 + a_2 z^2 + a_3 xy + a_4 yz + a_5 xz + a_6 x + a_7 y + a_8 z + a_9 = 0 
\label{eq: SurfaceConic}
\end{equation}
%
The parameter needed to define the correct surface conic shape that define uniquely the mirror desired are:
%
\begin{itemize}
\item focal distances
\item angle of incidence $\theta_g $, more precisely the program use the complementary angle of $\theta_g $ that is $\theta = \frac{\pi}{2} - \theta_g $ (the input and output angle are in radiant)
\end{itemize}
%
The conic surfaces are defined in such a way to have the origin equal to the incidence point of a collimated ray distant p (that correspond to the object focal distance) from the mirror, and with the normal of the surface corresponding to the z-axis, as it is showed in Figure \ref{fig: ellipse}.
For the plane mirror the situation is simple, because the equation of the surface is that in Equation \ref{eq: PlaneMirror}, that have all the coefficient equal to $0 $ apart from $a_8 $ that is equal to $1 $. 
%
\begin{equation}
z = 0
\label{eq: PlaneMirror}
\end{equation}
%
\\
For the spherical case, the parameter that characterize a sphere is the radius, one time defined the radius, the equation of the sphere is:
\begin{equation}
x^2 + y^2 + z^2 = r^2
\label{eq: sphere}
\end{equation}
Moreover, it is known, from the spherical lens optics (see Chapter \ref{capitolo2}) , that:
\begin{equation}
\frac{1}{p} + \frac{1}{q} = \frac{2}{r_t \sin \theta_g}
\label{eq: sphereLaw1}
\end{equation}
and
\begin{equation}
\frac{1}{p} + \frac{1}{q} = \frac{2 \sin \theta_g}{r_s}
\label{eq: sphereLaw2}
\end{equation}
where $r_t $, is the tangential radius, and $r_s $ is the saggital radius. The sphere case has $r_t = r_s $, this mean that, apart from the normal incidence case, the sphere cannot perfectly focalize/collimate a beam.The radius chosen in "$Surface\_$ $conic$" object is that corresponding to the equation \ref{eq: sphereLaw1}: 
\begin{equation}
r = \frac{2}{\sin \theta_g} \frac{pq}{p + q}
\end{equation}
where p correspond to the object focus length, q to the image lengths, $\theta_g $ to the incidence angle.
\\
For the paraboloid shape the correct coefficients that define the right surface are: the incidence angle, the focal distance and another parameter that distinguish between the cases shown in Figure \ref{fig: ParabolaFocus} and, Figure \ref{fig: ParabolaColl}. These two system correspond mirrors that, physically, have different behaviour, the first one Figure \ref{fig: ParabolaFocus}, focalize a Beam, the second one, Figure \ref{fig: ParabolaColl}, collimate a Beam.
\begin{figure}[H]
%
\centering
%
\subfloat[][Focalizing case\label{fig: ParabolaFocus}]
   {\includegraphics[width=.4\textwidth]{Immagini/Chapter3/ParabolaFoc}}\quad
%
\subfloat[][Collimating case\label{fig: ParabolaColl}]
   {\includegraphics[width=.4\textwidth]{Immagini/Chapter3/ParabolaColl}}
%
\caption{Parabola}
\label{fig :p3}
\end{figure}
The general equation of a parabola, such that in  Figure \ref{fig: ParabolaSystem} is
%
\begin{equation}
y = \frac{1}{4f} x^2
\label{eq: parabola}
\end{equation}
%
where $f $ is the focal distance of the parabola. After a few calculation, see Appendix \ref{cap:AppendixB}, it is possible to correlate $f $ with the input parameter:
%
\begin{equation}
f = d \sin^2 \theta
\label{eq: parabola2}
\end{equation}
%
where d is the object focal distance, in the case depicted in Figure \ref{fig: ParabolaFocus}, otherwise, in the case depicted in Figure \ref{fig: ParabolaColl}, d is the image focal distance.
%
\begin{figure}[H]
%
\centering
%
\includegraphics[width=0.8\textwidth]{Immagini/Chapter3/Parab}
%
\caption{Part of a parabola (blue), with other characteristic (different color) \cite{wikiParabola}}
%
\label{fig: ParabolaSystem}
%
\end{figure}
For the elliptical case the situation is represented in Figure \ref{fig: ellipse}. The general equation of an ellipse where appear two unknown $a$ and $b $.
%
\begin{equation}
\frac{x^2}{a^2} + \frac{y^2}{b^2} = 1
\label{eq: ellipse}
\end{equation}
%
It is possible to correlate (Appendix \ref{cap:AppendixB}), the focal distances plus the incidence angle with the two parameter$a$ and $b $ with the following two equations:
%
\begin{equation}
p = \frac{a + b}{2}
\label{eq: 1stEllipseEq}
\end{equation}
%
\begin{equation}
q = \sqrt{ab} \cos\theta
\label{eq: 2ndEllipseEq}
\end{equation}
%
\begin{figure}[H]
%
\centering
%
\includegraphics[width=.8\textwidth]{Immagini/Chapter3/EllipseSystem3}
%
\caption{Ellipse System}
%
\label{fig: ellipse}
%
\end{figure}
%
Once the surface is defined in the $x^{'}y^{'} $ reference frame, thean it is done a rotation and a translation in order to center it in the new $xy $ system centered on the point P. Normal at point $P $ is the new z-axis.
\\
For the hyperboloidal mirror the situation is similar to that of the ellipsoidal case, in fact, the general equation of the an hyperbola such the one in Figure is 
%
\begin{equation}
\frac{x^2}{a^2} - \frac{y^2}{b^2} = 1
\end{equation}
%
and the equations that correlate the focal distances and the incidence angle with the parameter $a $ and $b $ are
%
\begin{equation}
p = \frac{a - b}{2}
\label{eq: 1stHypEq}
\end{equation}
%
\begin{equation}
q = \sqrt{ab} \sin\theta
\label{eq: 2ndHypEq}
\end{equation}
%
\begin{figure}[H]
%
\centering
%
\includegraphics[width=.5\textwidth]{Immagini/Chapter3/Hyperbola}
%
\caption{Hyperbola System}
%
\label{fig: hyperbola}
%
\end{figure}
%
After that, as in the case of the ellipsoidal mirror, a rotation and a translation is needed to complete the work.
\\
In the program there is a further option that make the mirror cylindrical in one dimension maintaining its surface conic in the other. To do this, in "$Surface\_$ $conic object$", it is defined a "$set\_$ $cylindrical$" method, that change the shape of the surface, from a complete revolution conic, to a surface conic in one dimension and cylindrical in the other.
\\
In addition of the mirrors elements is implemented also an ideal lens element that follow the typical lens equation:
\begin{equation}
		\frac{1}{f_x} = \frac{1}{p} + \frac{1}{q}
		\label{eq: lens equation1}
\end{equation}
\begin{equation}
		\frac{1}{f_z} = \frac{1}{p} + \frac{1}{q}
		\label{eq: lens equation2}
\end{equation}
where $f_x $ is the x focal length and $f_z $ is the z focal length. For this optical element the input parameters are the object focal distance, image object distance and the two focal distances ($f_x $, $f_z $) that, in the default mode, are set equal with a value equal to $f_x = f_z = \frac{pq}{p+q} $.

\subsection{Compound Optical Element (KB and Montel system)}

This program include also two different system composed by more mirrors. Starting from conical mirrors, combining them, is possible to have a compound optical element that can simulate the behaviour of some typical instrumentation that characterize the facilities, in particular in the synchrotron world. The compound optical system implemented are two of those mentioned in Chapter \ref{capitolo2}
\begin{itemize}
\item Kirkpatrick-Baez system (KB system)
\item Montel
\end{itemize}

\subsection{Compound System: Kirkpatrick-Baez}
KirkpatrickBaez or, more simply, KB system is shown in Figure \ref{fig: KB}. It is composed by two cylindrical conic mirrors placed one after the other with the two focal lengths that converge in the same point. There are implemented two different kinds of KB system, a first one composed by two elliptical mirrors and a second one composed by parabolic mirrors. The input parameters that the program needs are the two incidence angles and the two foci, with respect to the centre of the KB system, represented in Figure \ref{fig: KB1} and the separation of the two mirror, from centre to centre.
\begin{figure}[H]
%
\centering
%
\includegraphics[width=.4\textwidth]{Immagini/Chapter3/KB}
%
\caption{KB system}
%
\label{fig: KB}
%
\end{figure}
\begin{figure}[H]
%
%\centering
%
\includegraphics[width=.8\textwidth]{Immagini/Chapter3/KB1}
%
\caption{Different object/image focal length that characterize the whole KB system, and the single mirrors. Is it also showed the separation length between the two mirror}
%
\label{fig: KB1}
%
\end{figure}
%%%%%%%%%%%%%%%%%%%%%%%%%%%%%5
%\begin{figure}[H]
%
%\centering
%
%\subfloat[][Focalizing case\label{fig: KB}]
%   {\includegraphics[width=.4\textwidth]{Immagini/Chapter4/KB}}\quad
%
%\subfloat[][Collimating case\label{fig: KB1}]
%   {\includegraphics[width=.4\textwidth]{Immagini/Chapter4/KB1}}
%
%\caption{KB system}
%\label{fig :p5}
%\end{figure}
This system is simply a system composed by two  conical mirrors in series. The parameter needed to define the mirror are not the ones defined by the user. The mirror parameters are the ones shown in Figure \ref{fig: KB1}: p1, q1, p2, q2. These parameters represent the object focal distance (p1, p2) and the image focal distances (q1, q2) of the two mirror, as represented in Figure \ref{fig: KB1}. Figure \ref{fig: CodeKB}, show an example of the definition for a KB system that have an object focal length of 2m, an image focal distance of 5m, a separation between the two centre  of the mirrors of 1m, and the two angle of incidence equal each other to $2^{\circ} $.
\begin{figure}[H]
%
\centering
%
\includegraphics[width=1.\textwidth]{Immagini/Chapter3/CodeKB}
%
\caption{Example 4}
%
\label{fig: CodeKB}
%
\end{figure}
\subsection{Compound System: Montel }
The Montel system, depicted in Figure \ref{fig: MontelSystem}, is composed, as for the KB, by two surface conical mirror cylindrical in one direction, but, because the two mirror are not in series, as for the case of the KB, the situation is a bit complicate. Starting from definition of the two mirrors one is rotated of 90$^\circ $, in order to have a mirror in the xy plane, and another one in the zy plane. As shown in Figure \ref{fig: MontelSystem} the centre of the Cartesian system is set in the point where the  optical axis of the system hit the compound system having the normal of the first normal equal to the z-axis, and the second normal equal to the - x-axis. The system is defined by the following parameters p, q, $\theta_z $, $\theta_x $, where p and q are the focal distances of the two mirrors and $\theta_z $ and $\theta_x $ are the angle of incidence to define the correct mirrors  (by default $\theta_z = \theta_x $).
\begin{figure}[H]
%
\centering
%
\includegraphics[width=1.\textwidth]{Immagini/Chapter3/MontelSystem2}
%
\caption{Grafical 3D view of the Montel in its reference system}
%
\label{fig: MontelSystem}
%
\end{figure}
The Figure \ref{fig: CodeMontel} shows an example code for a parabolic Montel system having an object focal length of 5m, image focal length of 2m and the two incidence angle of $1.5^{\circ} $, that focalize a Beam. As for the KB system also in this case there are implemented two possibilities, an ellipsoidal system (having the two mirror as ellipsoid), and parabolic system (having the two mirror as ellipsoid). 
\begin{figure}[H]
%
\centering
%
\includegraphics[width=1.\textwidth]{Immagini/Chapter3/CodeMontel}
%
\caption{Example 5}
%
\label{fig: CodeMontel}
%
\end{figure}
%
\section{The tracing mechanism}
Once defined the Beam and the different optical element, to complete a simulation, is needed a tool that put everything together and modify the property of the beam after the interaction with the optical elements.
\\
For example, if one want to simulate the system depicted in Figure \ref{fig: System}, one has to define a Beam source, the optical element and, at the end the distances between the optical elements. The tracing part of the program, for the non-compound optical element, is written in such a way that the trace work in series, one optical element after the other. This works with the definition of two distances, object/image distance from the centre  of the optical element, the incidence angle of the Beam. One possibility, to define the system in Figure \ref{fig: System} is to set the object distances of the mirrors equal to distance $d_0 $ and $d_1 $, and the image distances equal to $0 $, for the lens lets set the object distance equal to $d_3 $ and the image distance equal to $d_4 $ as it is reported in Figure \ref{fig: CodeSystem}.
\begin{figure}[H]
%
\centering
%
\includegraphics[width=1.0\textwidth]{Immagini/Chapter3/Cattura}
%
\caption{To simulate this system it need to define: source Beam parameters, optical elements and the distances between them}
%
\label{fig: System}
%
\end{figure}
\begin{figure}[H]
%
\centering
%
\includegraphics[width=1.\textwidth]{Immagini/Chapter3/CodeSystem}
%
\caption{Example 6}
%
\label{fig: CodeSystem}
%
\end{figure}
%
\subsection{Tracing for simple Optical element}
Going deeper in the code, the algorithm that trace a single element is divided in 5 steps:
\begin{enumerate}
	\item change the reference system from that of the optical axes to that of the optical element using two rotations, one along x-axis, and second along y-axis. Then a translation equal to the object distance of the optical element
	\item free propagation up to the optical element
	\item effect of the optical element (specular reflection)
	\item free propagation to the image plane
	\item changing the reference system to one that has the optical axis equal to the y-axis
\end{enumerate}
The first three points are condensed in the method "$effect\_$ $of\_$ $the\_$ $optical\_$ $element $", that is showed in Figure \ref{fig: CodeEffOpt}, and the last two point are condensed in the method "$effect\_$ $of\_$ $the\_$ $screen$" that is showed in Figure \ref{fig: CodeEffScreen}.
\begin{figure}[H]
%
\centering
%
\subfloat[][Example code for a Gaussian spot\label{fig: CodeEffOpt}]
   {\includegraphics[width=.4\textwidth]{Immagini/Chapter3/CodeOptElem}}\quad
%
\subfloat[][Plot of Figure \ref{fig: CodeGaussian}\label{fig: CodeEffScreen}]
   {\includegraphics[width=.4\textwidth]{Immagini/Chapter3/CodeOfScreen}}
%
\caption{Example 7}
\label{fig :p1}
%
\end{figure}
Because of the different definition, the tracing method of the rays' beam, need a different interpreter that can link the beam with the different optical elements that meet on his way. Because of the different nature, there are implemented two kinds of tracing, a first one that trace the KB system, that is composed by a series of optical elements and so can be used for all the compound optical elements that are in series. And a second one that is specific to the Montel system, because it is not composed by mirrors in series rather than mirrors in parallel, having the two elements in a very small region of the space that have in which order the rays of the beam hit the different mirrors.
\subsection{Tracing for KB}
For KB system the situation is more or less the same as for a simple optical mirrors, with the only difference that there are more than one mirror. So the algorithm to simulate the tracing system is nothing else than a for loop, that use the tracing system of the simple optical element. In this the object and the image distance from the centre  of the system are the default ones, such as for the incidence angles. Figure \ref{fig: CodeTraceCompund}, show the trace code for the compound elements that are in series.
\begin{figure}[H]
%
\centering
%
\includegraphics[width=1.\textwidth]{Immagini/Chapter3/CodeTraceCompound}
%
\caption{Tracing code for a compound optical element that use a sequential tracing method, such as KB system.}
%
\label{fig: CodeTraceCompund}
%
\end{figure}
\subsection{Tracing for Montel}
As it is said at the beginning of the Chapter, the tracing method used for Montel is new and different from the other. For the KB case the tracing method utilized is sequential. This mean that all the rays hit firstly a mirror, and secondly the other mirror. Figure \ref{fig : 111}, show the KB and Montel and situations respectively.
\begin{figure}[H]
%
\centering
%
\subfloat[][KB\label{fig: KB111}]
   {\includegraphics[width=.4\textwidth]{Immagini/Chapter3/KB111}}\quad
%
\subfloat[][Montel \label{fig: Montel111}]
   {\includegraphics[width=.4\textwidth]{Immagini/Chapter3/Montel111}}
%
\caption{For the KB system (\ref{fig: KB111}) all the rays hit sequentially one mirror after the other. For the Montel system (\ref{fig: Montel111}) a portion of rays coming from the source hit firstly oe1 and secondly oe2, the other portion do the opposite thing. From \cite{liu2011achromatic}.}
\label{fig : 111}
%
\end{figure}
\noindent The first step to trace the Montel is that to change the reference frame from the source optical axial to the optical system. This system is chosen to have origin in the centre  of whole system with the normal of the two mirror along z and -x axis, as it is showed in Figure \ref{fig: MontelSystem}. Moreover, the change of reference frame, has to be done in order to force the beam to have incidence angles, and distance from the centre  of the system equal to those chosen by the user. To do this, two rotations and one translation are needed.
\\
The second step is the propagation. The beam can hit any of the two mirrors a new method is implemented. This method, named "$comparison $ $\_ $ $time$", compares the time that each ray take to reach each mirror. Calculated the time for each ray, is set a flag that specify the closest optical element. At this point the Beam is divided in three "sub-Beams", a first one that contain the rays going to the first mirror, a second one that contain the rays going to the second mirror, and a third one that contain the rays that, because of the finite size of the mirrors, doesn't hit any mirror. At this point each sub-Beam is traced to the corresponding optical element and so are computed the effect of the mirror on the 1st and 2nd sub-Beams. This process gives three output: "beam1", "beam2" and "beam3". These output contain the information of the beam along the whole propagation system.
\\
%\begin{figure}[H]
%
%\centering
%
%\includegraphics[width=1.\textwidth]{Immagini/Chapter3/Montel_x_ray_beam}
%
%\caption{Example 8}
%
%\label{fig: RealMontel}
%
%\end{figure}
The "beam1" and "beam2", are list of two dimension that contain the rays of that will hit the first mirror at the first and at the second iteration. "beam3" is the most important output, because contain the information of the beam at the image plane. This list is divided in three Beam, where the first beam contain the information of the ray that undergo no-reflection, the second Beam contain the information of the ray that undergo one-reflection and the third contain the information of the ray that undergo two-reflections.
Figure \ref{fig: BigSourceMontel}, show the image produced by the 3 Beams. The red dots correspond to the beam with no-reflection, the blue dots correspond to the beam with a single-reflection and the green dots correspond to the beam with two-reflection.
\\
The last step of this tracing method consist in a re-change of reference frame to the optical axis of the two reflected beam (as showed in Figure \ref{fig: BigSourceMontel}, where the green dots are centred in the point (0,0)). This transformation is similar to the first step of this tracing method where the two reflection are done with respect to the same angle of incidence, and the translation depend on the value set by the user corresponding to the distance from the centre  of the Montel system.
\\
Here below a brief summary of the tracing method:
	\\ \hspace{20mm} 1. Changing the reference frame from the source optical axis to the Montel system
	\\ \hspace{20mm} 2. Focus the attention on the travel time of each ray in order to know which is the nearest optical element of each ray
	\\ \hspace{20mm} 3. free propagation of each ray up to the nearest optical element
	\\ \hspace{20mm} 4. effect of the system for each ray
	\\ \hspace{20mm} 5. repeat the 2$^{nd} $, 3$^{rd} $ and 4$^{th} $ passage two times, in order to consider the two reflection
	\\ \hspace{20mm} 6. Change the reference system to the optical axes that is subject to two reflection, doing two rotation and one translation
\begin{figure}[H]
%
\centering
%
\includegraphics[width=1.\textwidth]{Immagini/Chapter3/BigSourceMontel}
%
\caption{Example 8}
%
\label{fig: BigSourceMontel}
%
\end{figure}
\noindent Because of the importance of the Montel system in my Thesis, it is implemented some additional options with the aim to study better the behaviour of the system. These new parameters that affect the tracing system are: 
	\\ \hspace{20mm} 1. focal distances and incidence angles, that define the two rotation and the translation of the tracing system
	\\ \hspace{20mm} 2. name of the File in which is saved the data of the simulation, by default no data is saved
	\\ \hspace{20mm} 3. there is the possibility to choose a different point, from the origin, in which the optical axis hit the system
	\\ \hspace{20mm} 4. there is also the possibility to have a final output frame that doesn't correspond to the two-reflected beam, but with the non reflected beam or with the other two beam that are reflected only one time
	\\ \hspace{20mm} 5. It is also the possibility to figure out the footprint of the two reflected beam on the system. For clarity the beam that hit the first mirror and after the second is labelled with red point, the beam that hit the second and after first mirror is labelled by blue colour.

\noindent In Chapter \ref{capitolo4}, these option will be useful in the comprehension of the Montel. The possibilities to change the angle of incidence and to hit different part from the origin can be used to study what happen to a beam when is not aligned, or not perfectly aligned, and use these result to align the system in the laboratories. The possibilities to save a File is useful in particular in those case where there is a huge computational effort that need a lot of time, in these cases is possible to work with the result of a big simulation without reappointing it, and so save time. Figure \ref{fig: CodeTraceMontel} show the code that trace a Montel elements, containing also the special option that were defined above.
\begin{figure}[H]
%
\centering
%
\includegraphics[width=.65\textwidth]{Immagini/Chapter3/CodeTraceMontel}
%
\caption{Example 9}
%
\label{fig: CodeTraceMontel}
%
\end{figure}