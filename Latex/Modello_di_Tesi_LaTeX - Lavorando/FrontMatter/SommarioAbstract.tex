% ------------------------------------------------------------------------ %
% !TEX encoding = UTF-8 Unicode
% !TEX TS-program = pdflatex
% !TEX root = ../Tesi.tex
% !TEX spellcheck = it-IT
% ------------------------------------------------------------------------ %
%
% ------------------------------------------------------------------------ %
% 	SOMMARIO + ABSTRACT
% ------------------------------------------------------------------------ %
%
\cleardoublepage
%
\phantomsection
%
\pdfbookmark{Sommario}{Sommario}
%
% ------------------------------------------------------------------------ %
% consento presenza di più capitoli nella stessa pagina
\begingroup
%\let\clearpage\relax
\let\cleardoublepage\relax
\let\cleardoublepage\relax
% ------------------------------------------------------------------------ %
%
\chapter*{Sommario}
%
La simulazione delle proprietà di fasci di raggi X lungo la propagazione nelle beamline, è uno step importante per la progettazione e l'ottimizzazione di esse. Per sistemi ottici basati sulla riflessione singoli specchi o particolari combinazioni di specchi vengono utilizzati per la focalizzazione del fascio. Un esempio tipico di questi sistemi è il sistema KirckPatrick-Baez (KB), molto poopolare all'ESRF per le sue molteplici qualità. Comunque, siccome la qualità dei fasci generata dal sincrotrone dell'ESRF è molto elevate, si va sempre più alla ricerca di elementi ottici via via migliori.
\\
Durante il mio periodo di tirocinio all'ESRF, ho sviluppato una libreria python, in grado di simulare la propagazione di fasci attraverso semplici specchi sferici, e combinazioni di essi quali il sopracitato KirckPatrick-Baez e un altro sistema, denominato Montel. Lo scopo della tesi è stato quello di studare il sistema Montel utilizzando la libreria generata.
\medskip
%
%\noindent \textbf{Parole chiave:} 
%PoliMi,
%Tesi,
%LaTeX,
%Scribd
%
\clearpage
%\vfill
%
% ------------------------------------------------------------------------ %
%
\selectlanguage{english}
%
\pdfbookmark{Abstract}{Abstract}
%
\chapter*{Abstract}
%Text of the abstract in english
The simulation of x-ray beam properties during the transport along a beamline is important for the design, the optimization and the operation of the beamline. For reflection optical system single mirrors or particular combination of them are used to increase the focusing property of a beam. A typical example is the Kirkpatrick-Baez (KB) system, very popular at the ESRF because of its many good properties and so well studied. However, the extreme quality of the synchrotron beams that will be available with the ESRF upgraded storage ring pushes the requirement in optics to consider more and more perfect elements.

During my trainership period at ESRF, I developed a python library in able to simulate a beam propagation along simple surface conic mirrors, and combination of them such as the already discussed KirckPatrick-Baez and another system, named Montel. The aim of the thesis was that to study the effect of Montel system using the builded library. 
\medskip
%
%\noindent \textbf{Keywords:} 
%PoliMi,
%Master Thesis,
%LaTeX,
%Scribd
%
% ------------------------------------------------------------------------ %
%
\selectlanguage{italian}
%
\endgroup			
%
%\vfill
%
% ------------------------------------------------------------------------ %