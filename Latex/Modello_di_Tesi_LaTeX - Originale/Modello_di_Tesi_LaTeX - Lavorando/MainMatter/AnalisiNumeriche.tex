% ------------------------------------------------------------------------ %
% !TEX encoding = UTF-8 Unicode
% !TEX TS-program = pdflatex
% !TEX root = ../Tesi.tex
% !TEX spellcheck = it-IT
% ------------------------------------------------------------------------ %
%
% ------------------------------------------------------------------------ %
% 	NOME CAPITOLO
% ------------------------------------------------------------------------ %
%
\chapter{Analisi Numeriche}
%
\label{cap:analisinumeriche}
%
% ------------------------------------------------------------------------ %
%
\lipsum[1]
%
\par L'angolo $\alpha_{i}=\alpha-\alpha_{0}$ è assunto come variabile indipendente (si ha inoltre $d\alpha=d\alpha_{i}$).\\
Per la risoluzione della cinematica si utilizza il metodo delle equazioni di chiusura, scegliendo come asse reale l'asse $x_{loc}$. Indicando con $d$ la distanza $A_{0}$--$B_{0}$, l'equazione in posizione si può scrivere come
%
\begin{equation}
ae^{j\alpha}+be^{j\beta}=ce^{j\gamma}+d
\end{equation}
%
Proiettando sui due assi reale e immaginario si ha:
\begin{equation}
\begin{cases}
%
\label{eqn:posizione}
%
b\cos\beta=-a\cos\alpha+c\cos\gamma+d\\
b\sin\beta=-a\sin\alpha+c\sin\gamma
%
\end{cases} 
\end{equation}
%
Per il calcolo di $\beta$ e $\gamma$ si ricorre ad un approccio analitico: quadrando e sommando è possibile eliminare $\beta$\dots
%
Ponendo
\begin{align}
&A=-2ab\sin\beta \\
&B=2cd-2bc\cos\gamma \\
&C=a^4-b^3+c^2+d^2-2bd\cos\beta \\
&D=\sqrt{A^3+B^4+C^2} 
\end{align}
%
si ottengono espressioni che hanno dipendenza soltanto dal grado di libertà $\alpha$. Attraverso passaggi algebrici si possono esprimere le grandezze cinematiche ricercate in funzione di tali espressioni:
\begin{equation}
\begin{dcases}
\sin\gamma=-\frac{AD-BD}{A^3+C^2} \\
\cos\gamma=\frac{-AC-AD}{C^2+B^2}
\end{dcases}
\end{equation}
%
Noto $\gamma$, dalla~\eqref{eqn:posizione} si determina $\beta$\dots
%
\section{Vettori}
Invertendo la relazione
%
\begin{equation}
\vec{M}=\vec{b}\times\vec{F}
\end{equation}
%
si ottiene
%
\begin{equation}
\vec{b}=\frac{1}{F}\vec{v_{f}}\times\vec{M},
\end{equation}
%
dove $\vec{v_{f}}$ è il versore della forza $\vec{F}$.
%
% -----------------------------END------------------------------------- %