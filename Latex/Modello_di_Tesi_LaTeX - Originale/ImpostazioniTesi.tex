% ------------------------------------------------------------------------ %
% !TEX encoding = UTF-8 Unicode
% !TEX TS-program = pdflatex
% !TEX root = Tesi.tex
% !TEX spellcheck = it-IT
% ------------------------------------------------------------------------ %
%
% ------------------------------------------------------------------------ %
% 	PREAMBOLO - SETUP
% ------------------------------------------------------------------------ %
% Comandi personali
% ------------------------------------------------------------------------ %
%
\newcommand{\myName}{Luca Maggiori}			% autore
\newcommand{\myMatricola}{783186}			% matricola
\newcommand{\myTitle}{Modello di Tesi di Laurea in \LaTeX{}}	% titolo
\newcommand{\myUni}{Politecnico di Milano}		% università
\newcommand{\myFaculty}{Scuola di Ingegneria Industriale e dell'Informazione}	% facoltà/scuola
\newcommand{\myDegree}{Ingegneria Meccanica}		% laurea
\newcommand{\myThesis}{Tesi di Laurea Magistrale}	% tipo di tesi
\newcommand{\myDepartment}{Dipartimento di Meccanica}	% dipartimento
\newcommand{\myProf}{Prof.~Charles~Dickens}		% relatore
\newcommand{\myOtherProf}{Ing.~Emilio~Salgari}		% eventuale correlatore
\newcommand{\myLocation}{Milano}			% dove
\newcommand{\myTime}{Aprile 2014}			% quando
\newcommand{\myAcademicYear}{2012--2013}		% anno accademico
\newcommand{\myLogo}{logoPoliMi}			% logo
\newcommand{\myLogoCFD}{logoPoliMiCFD}		% logo CFD :-)
\newcommand{\myUrlUni}{www.polimi.it}			% sito PoliMi
\newcommand{\myUrlFaculty}{www.ingindinf.polimi.it}	% sito Facoltà
%
% ------------------------------------------------------------------------ %
% Impostazioni di amsmath, amssymb, amsthm
% ------------------------------------------------------------------------ %
%
% un ambiente per i sistemi
\newenvironment{sistema}%
	{\left\lbrace\begin{array}{@{}l@{}}}%
	{\end{array}\right.}
%
% epsilon theta rho phi
\renewcommand{\epsilon}{\varepsilon}
\renewcommand{\theta}{\vartheta}
%\renewcommand{\rho}{\varrho}
\renewcommand{\phi}{\varphi}
%
\renewcommand{\vec}{\mathbf} 	% vettori in tondo nero
%
% ------------------------------------------------------------------------ %
% Impostazioni di biblatex
% ------------------------------------------------------------------------ %
%
% -- commentare o cancellare tutto se si desidera bibliografia standard
%
% I comandi seguenti saranno poi usati in Bibliografia.tex
% per suddividere i riferimenti bibliografici tra materiale citato
% e materiale non citato nel testo, con l'ulteriore distinzione in
% materiale cartaceo e materiale online (con link)
%
% Al termine si riportano anche pubblicazioni legate a Latex
% e alla stesura della tesi di laurea
%
\newcommand{\bibtitolocitati}{Riferimenti citati nel testo}
\newcommand{\bibtitolocitaticarta}{Pubblicazioni e Manuali}
\newcommand{\bibtitolocitatiweb}{Materiale Online}
\newcommand{\bibtitolononcitati}{Ulteriore materiale consultato}
\newcommand{\bibtitolononcitaticarta}{Pubblicazioni e Manuali}
\newcommand{\bibtitolononcitatiweb}{Materiale Online}
\newcommand{\bibtitololatex}{{\LaTeX{}}}
%
\DeclareBibliographyCategory{citati}
%
\defbibheading{citati-cartacei}{\subsection*{\bibtitolocitaticarta}}
\defbibheading{citati-web}{\subsection*{\bibtitolocitatiweb}}
\defbibheading{non-citati}{\section*{\bibtitolononcitati}}
\defbibheading{non-citati-cartacei}{\subsection*{\bibtitolononcitaticarta}}
\defbibheading{non-citati-web}{\subsection*{\bibtitolononcitatiweb}}
\defbibheading{latex}{\subsection*{\bibtitololatex}}
%
\AtEveryCitekey{\addtocategory{citati}{\thefield{entrykey}}}
%
\AtEveryBibitem{
    \clearfield{doi}
    \clearfield{eprint}
}
%
\nocite{*}	% manda in bibliografia anche tutte le opere non citate
%
% ------------------------------------------------------------------------ %
%
% Decommentare i comandi che seguono
% se si vuole ripristinare bibliografia standard
% (commentando tutto il blocco precedente)
%
%\defbibheading{bibliography}{%
%	\cleardoublepage%
%	\phantomsection%
%	\addcontentsline{toc}{chapter}{\bibname}%
%	\chapter*{\bibname\markboth{\bibname}{\bibname}}%
%	}
%
% ------------------------------------------------------------------------ %
% Impostazioni di xcolor
% ------------------------------------------------------------------------ %
%
% webcolors
\definecolor{webgreen}{rgb}{0,.5,0}
\definecolor{webbrown}{rgb}{.6,0,0}
%
% BluePolimi (colori delle presentazioni PPT del Politecnico di Milano)
\definecolor{darkbluePoliMi}{rgb}{0,0.18,0.40}	%rgb(0, 46, 103)
\definecolor{midbluePoliMi}{rgb}{0.33,0.47,0.62}	%rgb(84, 121, 157)
\definecolor{lightbluePoliMi}{rgb}{0.53,0.64,0.73}	%rgb(134, 163, 186)
\definecolor{orangePoliMi}{rgb}{1,0.59,0}		%rgb(255, 151, 0)
%
% redSapienza (rosso Sapienza)
\definecolor{redSapienza}{rgb}{0.514,0.031,0.165}	%rgb(131, 8, 42)
%
% ------------------------------------------------------------------------ %
% Impostazioni di listings
% ------------------------------------------------------------------------ %
%
\lstset{
	basicstyle=\smaller[0]\ttfamily,		% Black & White:
	keywordstyle=\color{RoyalBlue},	% keywordstyle=\color{black}\bfseries,
	commentstyle=\color{webgreen},	% commentstyle=\color{gray},
	stringstyle=\color{webbrown},		% stringstyle=\color{black},
	numbers=left,
	numberstyle=\smaller[2],
	stepnumber=1,
	numbersep=8pt,
	showspaces=false,
	showstringspaces=false,
	showtabs=false,
	breaklines=true,
	frameround=ffff,
	frame=single,
	tabsize=2,
	captionpos=t,
	breakatwhitespace=false,
	}
%
% Solution to the encoding issue
\lstset{literate=
  {á}{{\'a}}1 {é}{{\'e}}1 {í}{{\'i}}1 {ó}{{\'o}}1 {ú}{{\'u}}1
  {Á}{{\'A}}1 {É}{{\'E}}1 {Í}{{\'I}}1 {Ó}{{\'O}}1 {Ú}{{\'U}}1
  {à}{{\`a}}1 {è}{{\`e}}1 {ì}{{\`i}}1 {ò}{{\`o}}1 {ù}{{\`u}}1
  {À}{{\`A}}1 {È}{{\'E}}1 {Ì}{{\`I}}1 {Ò}{{\`O}}1 {Ù}{{\`U}}1
  {ä}{{\"a}}1 {ë}{{\"e}}1 {ï}{{\"i}}1 {ö}{{\"o}}1 {ü}{{\"u}}1
  {Ä}{{\"A}}1 {Ë}{{\"E}}1 {Ï}{{\"I}}1 {Ö}{{\"O}}1 {Ü}{{\"U}}1
  {â}{{\^a}}1 {ê}{{\^e}}1 {î}{{\^i}}1 {ô}{{\^o}}1 {û}{{\^u}}1
  {Â}{{\^A}}1 {Ê}{{\^E}}1 {Î}{{\^I}}1 {Ô}{{\^O}}1 {Û}{{\^U}}1
  {œ}{{\oe}}1 {Œ}{{\OE}}1 {æ}{{\ae}}1 {Æ}{{\AE}}1 {ß}{{\ss}}1
  {ç}{{\c c}}1 {Ç}{{\c C}}1 {ø}{{\o}}1 {å}{{\r a}}1 {Å}{{\r A}}1
  {€}{{\EUR}}1 {£}{{\pounds}}1
}
%
% Definizione ambienti per i vari linguaggi
%
\lstnewenvironment{Matlab}{\lstset{language=Matlab}}{}
%
\lstnewenvironment{C++}{\lstset{language=C++}}{}
%
\lstnewenvironment{bash}{\lstset{language=bash}}{}
%
%
% Comando per dare nome alla lista dei codici
%
\addto\captionsitalian{\renewcommand{\lstlistingname}{Codice}}
%
\addto\captionsitalian{\renewcommand{\lstlistlistingname}{Elenco dei codici}}
%
%\renewcommand{\lstlistingname}{Elenco dei codici}
%\renewcommand{\lstlistlistingname}{\lstlistingname}
%
% ------------------------------------------------------------------------ %
% Impostazioni di hyperref
% ------------------------------------------------------------------------ %
%
% per la descrizione delle varie opzioni vedere
% la guida del pacchetto hyperref
%
\hypersetup{
	%hyperfootnotes=false,
	%plainpages=false,
	%pdfpagelabels,
	colorlinks=true,
	linktocpage=true,	% true=link nei numeri pagina / false=link nel titolo
	pdfstartpage=1,
	pdfstartview=FitV,
	breaklinks=true,
	pageanchor=true,
	pdfpagemode=UseOutlines,
	%bookmarksnumbered,
	%bookmarksopen=true,
	bookmarksopenlevel=1,
	hypertexnames=true,
	pdfhighlight=/O,
	urlcolor=webbrown,		% colore dei link a pagine web
	linkcolor=RoyalBlue,		% colore dei collegamenti nel testo
	citecolor=webgreen,		% colore delle citazioni
	pdftitle={\myTitle},		% da qui in poi compilazione metadati
	pdfauthor={\textcopyright\ \myName, \myUni},
	pdfsubject={},
	pdfcreator={pdfLaTeX},
	pdfproducer={LaTeX with hyperref},
	pdfkeywords={polimi,
		tesi,
		latex,
		laurea,
		dottorato,
		scribd},
}
%
% comando per inviare mail
\newcommand{\mail}[1]{\href{mailto:#1}{\texttt{#1}}}
%
% Si possono avere tutti i collegamenti in nero e senza riquadri
% scrivendo semplicemente:
% \hypersetup{hidelinks}
%
% ------------------------------------------------------------------------ %
% Impostazioni di graphicx
% ------------------------------------------------------------------------ %
%
% Elenco dei percorsi in cui saranno cercate le immagini da inserire
%
% In questo modo non è necessario specificare il percorso relativo
% dell'immagine all'interno di \includegraphics{}, ma solo il nome.
%
% N.B. assicurarsi che non siano presenti più immagini
% con lo stesso nome.
%
\graphicspath{
	{Immagini/}
	{Immagini/Introduzione/}
	{Immagini/ProveSperimentali/}
	{Immagini/ProveSperimentali/Subfolder1/}
	{Immagini/ProveSperimentali/Subfolder2/}
	{Immagini/AnalisiNumeriche/}
	}
%
% ------------------------------------------------------------------------ %
% Impostazioni di caption
% ------------------------------------------------------------------------ %
%
\captionsetup{tableposition=top,
	figureposition=bottom,
	font=small,
	format=hang,
	labelfont=bf}
%
% ------------------------------------------------------------------------ %
% Impostazioni di fancyhdr
% ------------------------------------------------------------------------ %
%
% Impostazioni preferibili, ma NON del tutto adeguate alle norme POLIMI
% N.B. si possono usare queste impostazioni senza problemi anche per il PoliMi.
%
\pagestyle{fancy}			% sostituisce \pagestyle{header} standard
%
%\renewcommand{\chaptermark}[1]{	% ridefinisce indicazione capitolo
%	\markboth{\chaptername\ \thechapter.\ #1}{}}
%
\makeatletter 			% necessary for using \@chapapp
\renewcommand{\chaptermark}[1]{	% ridefinisce indicazione capitolo
  \markboth{\@chapapp\ \thechapter.\ #1}{}} % distinzione 'Capitolo' / 'Appendice'
\makeatother
%
\renewcommand{\sectionmark}[1]{	% ridefinisce indicazione sezione
	\markright{\thesection.\ #1}}
%
\fancyhf{}				% svuota testatine e piede
%
\fancyhead[LE,RO]{\bfseries\thepage}	% numero pagine in alto
%
\fancyhead[LO]{\bfseries\rightmark}	% info sezione nelle pag. dispari
%
\fancyhead[RE]{\bfseries\leftmark}	% info capitolo nelle pag.pari
%
\renewcommand{\headrulewidth}{0.4pt}	% spessore linea header
%
\renewcommand{\footrulewidth}{0pt}	% spessore linea footer (0pt=nascosta)
%
\fancypagestyle{plain}{				% ridefinizione stile inizio capitolo
		\fancyhead{}			% header vuoto
		\fancyfoot[C]{\bfseries\thepage}		% numeri in grassetto al centro
		\renewcommand{\headrulewidth}{0pt}	% no linea
		}
%
% ------------------------------------------------------------------------ %
%
% Impostazioni maggiormente in linea con le norme POLIMI
% (decommentare l'intero blocco e commentare il blocco precedente)
%
% N.B. si possono usare le impostazioni precedenti senza problemi anche
% per il PoliMi (e infatti io ho usato le precedenti). Il mio consiglio è di
% usare, tra le 2 versioni proposte, quella sopra.
%
%\pagestyle{fancy}			% sostituisce \pagestyle{header} standard
%%
%\makeatletter 			% necessary for using \@chapapp
%\renewcommand{\chaptermark}[1]{%
%  \markboth{\@chapapp\ \thechapter.\ #1}{}} % distinzione 'Capitolo' / 'Appendice'
%\makeatother
%%
%\fancyhf{}				% svuota testatine e piede
%%
%\fancyfoot[LE,RO]{\bfseries\thepage}	% numero pagine in basso
%%
%\fancyhead[RO]{\bfseries\leftmark}	% info capitolo pagine dispari
%%
%\fancyhead[LE]{\bfseries\leftmark}	% info capitolo pagine pari
%%
%\renewcommand{\headrulewidth}{0.4pt}	% spessore linea header
%%
%\renewcommand{\footrulewidth}{0pt}	% spessore linea footer (0pt=nascosta)
%%
%\fancypagestyle{plain}{				% ridefinizione stile inizio capitolo
%		\fancyhf{}				% header e footer azzerati
%		\fancyfoot[C]{\bfseries\thepage}		% numero di pagina al centro
%		\renewcommand{\headrulewidth}{0pt}	% no linea header
%		}
%
% ------------------------------------------------------------------------ %
% Impostazioni degli acronimi
% ------------------------------------------------------------------------ %
%
% descrizione acronimi GIUSTIFICATA
\makeatletter
\def\bflabel#1{{\textbf{\textsf{#1}}\hfill}}
\renewenvironment{AC@deflist}[1]%
{\ifAC@nolist%
\else%
\begin{list}{}%
{\settowidth{\labelwidth}{\textbf{\textsf{#1}}}%
\setlength{\leftmargin}{\labelwidth}%
\addtolength{\leftmargin}{\labelsep}%
\renewcommand{\makelabel}{\bflabel}}%
\fi}%
{\ifAC@nolist%
\else%
\end{list}%
\fi}%
\makeatother
%
% ------------------------------------------------------------------------ %
% Altro
% ------------------------------------------------------------------------ %
%
% Gradiente
\newcommand{\gradiente}[1]{$\nabla #1$}
%
% puntini di omissione [...]
\newcommand{\omissis}{[\dots\negthinspace]}
%
% Eccezioni all'algoritmo di sillabazione
\hyphenation{OpenFOAM}
\hyphenation{Matlab}
\hyphenation{bash}
%
% ------------------------------------------------------------------------ %
% Finezze tipografiche per il Politecnico di Milano
% ------------------------------------------------------------------------ %
%
% Le seguenti modifiche possono essere commentate
% o adeguate ad un'altra università (es. 'Yale Blue'
% per l'università di Yale, 'Rosso Sapienza' per La Sapienza..)
%
% Filetti tabelle colorati
\arrayrulecolor{darkbluePoliMi}
%
%
% Righe delle note a piè di pagina colorate
\renewcommand{\footnoterule}{%
  \kern -3pt
  {\color{darkbluePoliMi} \hrule width 0.4\textwidth}
  \kern 2.6pt
}
%
% ------------------------------------------------------------------------ %